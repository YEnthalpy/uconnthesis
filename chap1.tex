%%%
%%% This is the beginning of the actual thesis.  If you don't know latex
%%% then start with the LaTeX manual by Lamport and another easy
%%% reference, like the paperback by Jane Hahn, LaTeX for Everyone, PTI,
%%% 1991. See also $TEX/latex/sample.tex and $TEX/doc/story.tex, where
%%% $TEX==/usr/local/lib/tex
%%%
% Start this dissertation....
%
\chapter{Introduction}\label{introduction}   % level 1
%
%% Stolen from the sample.tex file.  There have been a few
%% modifications to fit in the thesis here.
%
% This is a sample LaTeX input file.  (Version of 28 May 1985)
%
% A '%' character causes TeX to ignore all remaining text on the line,
% and is used for comments like this one.
%
% \author{Leslie Lamport} % For this section Lamport is the author.
% \title{A Sample Document}
% \date{December 12, 1984}
%
%
%\fixchapterheading % Use this if section follows chapter immediately
\section{The Sample.tex file}  % Produces section heading.  % level 2
%
    % Lower -level sections are begun with similar
    % \subsection and \subsubsection commands.


\subsection{Ordinary Text}   % level 3

The ends of words and sentences are marked by spaces. It doesn't matter
how many spaces you type; one is as good as 100.  The end of a line
counts as a space.\footnote{
This is a sample input file.  Comparing it with the output it
generates can show you how to produce a simple document of
your own.
}

One or more blank lines denote the end of a paragraph.

Since any number of consecutive spaces are treated like a single one,
the formatting of the input file makes no difference to \LaTeX,
but it makes a difference to you. When you use
\LaTeX,       % The \LaTeX command generates the LaTeX logo.
making your input file as easy to read as possible will be a great help
as you write your document and when you change it.  This sample file
shows how you can add comments to your own input file.

Because printing is different from typewriting, there are a number of
things that you have to do differently when preparing an input file than
if you were just typing the document directly.
Quotation marks like
       ``this''
have to be handled specially, as do quotes within quotes:
       ``\,`this'                  % \, separates the double and single quote.
        is what I just
        wrote, not  `that'\,.''

Dashes come in three sizes: an
       intra-word
dash, a medium dash for number ranges like
       1--2,
and a punctuation
       dash---like
this.

A sentence-ending space should be larger than the space between words
within a sentence.  You sometimes have to type special commands in
conjunction with punctuation characters to get this right, as in the
following sentence.
       Gnats, gnus, etc.\    % `\ ' makes an inter-word space.
       all begin with G\@.   % \@ marks end-of-sentence punctuation.
You should check the spaces after periods when reading your output to
make sure you haven't forgotten any special cases.
Generating an ellipsis
       \ldots\    % `\ ' needed because TeX ignores spaces after
                  % command names like \ldots made from \ + letters.
                  %
                  % Note how a `%' character causes TeX to ignore the
                  % end of the input line, so these blank lines do not
                  % start a new paragraph.
with the right spacing around the periods
requires a special  command.

\LaTeX\ interprets some common characters as commands, so you must type
special commands to generate them.  These characters include the
following:
       \$ \& \% \# \{ and \}.

In printing, text is emphasized by using an
       {\em italic\/}  % The \/ command produces the tiny extra space that
                       % should be added between a slanted and a following
                       % unslanted letter.
type style.

\begin{em}
   A long segment of text can also be emphasized in this way.  Text within
   such a segment given additional emphasis
          with\/ {\em Roman}
   type.  Italic type loses its ability to emphasize and become simply
   distracting when used excessively.
\end{em}

It is sometimes necessary to prevent \LaTeX\ from breaking a line where
it might otherwise do so.  This may be at a space, as between the
``Mr.'' and ``Jones'' in
       ``Mr.~Jones,''        % ~ produces an unbreakable interword space.
or within a word---especially when the word is a symbol like
       \mbox{\em itemnum\/}
that makes little sense when hyphenated across
       lines.

Footnotes\footnote{This is an example of a footnote.}
pose no problem.

\LaTeX\ is good at typesetting mathematical formulas like
       $ x-3y = 7 $
or
       $$ a_{1} > x^{2n} / y^{2n} > x'. $$
Remember that a letter like
       $x$        % $ ... $  and  \( ... \)  are equivalent
is a formula when it denotes a mathematical symbol, and should
be treated as one.

\subsection{Displayed Text}

Text is displayed by indenting it from the left margin.

\subsubsection{Quotations}

Quotations are commonly displayed.  There are short quotations
\begin{quote}
   This is a short a quotation.  It consists of a
   single paragraph of text.  There is no paragraph
   indentation.
\end{quote}
and longer ones.
\begin{quotation}
   This is a longer quotation.  It consists of two paragraphs
   of text.  The beginning of each paragraph is indicated
   by an extra indentation.

   This is the second paragraph of the quotation.  It is just
   as dull as the first paragraph.
\end{quotation}

\subsubsection{Lists}

Another frequently-displayed structure is a list.

\paragraph{Itemize.}
The following is an example of an {\em itemized} list.

%\minusline % Part of uuthesis.sty to remove extra vertical space.

\begin{quote}
\begin{itemize}
   \item  This is the first item of an itemized list.  Each item
          in the list is marked with a ``tick''.  The document
          style determines what kind of tick mark is used.

   \item  This is the second item of the list.  It contains another
          list nested inside it.  The inner list is an {\em enumerated}
          list.
          \begin{enumerate}
              \item This is the first item of an enumerated list that
                    is nested within the itemized list.

              \item This is the second item of the inner list.  \LaTeX\
                    allows you to nest lists deeper than you really should.
          \end{enumerate}
          This is the rest of the second item of the outer list.  It
          is no more interesting than any other part of the item.
   \item  This is the third item of the list.
\end{itemize}
\end{quote}

\paragraph{Verse.}
You can even display poetry.

%\minusline % Part of uuthesis.sty to kill one line

\begin{quote}
\begin{quote}
\begin{verse}
   There is an environment for verse \\    % The \\ command separates lines
   Whose features some poets will curse.   % within a stanza.

                           % One or more blank lines separate stanzas.

   For instead of making\\
   Them do {\em all\/} line breaking, \\
   It allows them to put too many words on a line when they'd
   rather be forced to be terse.
\end{verse}
\end{quote}
\end{quote}

\subsubsection{Mathematics}
Mathematical formulas may also be displayed.  A displayed formula is
one-line long; multiline formulas require special formatting
instructions.
   \[  x' + y^{2} = z_{i}^{2}\]
Don't start a paragraph with a displayed equation, nor make
one a paragraph by itself.

\section{More examples: Jeff McGough's Thesis}

Equations like
$\gamma = 0$ that don't need numbering may
be
set inline by the coding \verb"$\gamma = 0$" or displayed by
\par
%\begin{singlespace}
\begin{verbatim}
$$
\gamma = 0.
$$
\end{verbatim}
%\end{singlespace}
\par
Numbered equations are set as shown in the next paragraph. They use the
theorem environments defined in \verb"thesis.sty":
\par
%\begin{singlespace}
\begin{verbatim}
\newtheorem{theorem}{Theorem}[chapter]
\newtheorem{assertion}[theorem]{Assertion}
\newtheorem{claim}[theorem]{Claim}
\newtheorem{conjecture}[theorem]{Conjecture}
\newtheorem{corollary}[theorem]{Corollary}
\newtheorem{definition}[theorem]{Definition}
\newtheorem{example}[theorem]{Example}
\newtheorem{figger}[theorem]{Figure}
\newtheorem{lemma}[theorem]{Lemma}
\newtheorem{prop}[theorem]{Proposition}
\newtheorem{remark}[theorem]{Remark}
\end{verbatim}
%\end{singlespace}
\par

The Gelfand problem is the following elliptic boundary value problem:
%
% The equation-array feature in LaTeX is a bad idea.  For centered
% numbers you should set your own equations and arrays as follows:
%
\def\dd{\displaystyle}
\begin{equation}\label{gelfand}
\begin{array}{rl}
\dd \Delta u + \lambda e^u = 0, &
\dd u\in \Omega,\\[8pt] % add 8pt extra vertical space. 1 line=23pt
\dd u=0, & \dd u\in\partial\Omega.
\end{array}
\end{equation}
The previous equation had a label.  It may be referenced as
equation~(\ref{gelfand}).

%
%
\section{History of the Gelfand problem}
%
%

According to Bebernes and Eberly \cite[p.46]{bebernes:mpc89},
Gelfand was ``the first to make an in-depth
study'' of (\ref{gelfand}). Following this statement they briefly
outline the history of the Gelfand problem.
\par
% Quotes need to forced single space:
%\begin{singlespace}
\begin{quote}
For dimension $n=1$, Liouville~\cite{liouville:edp53} first studied and
found an explicit solution in 1853. For $n=2$, Bratu~\cite{bratu:ein14}
found an explicit solution in 1914.  Frank-Kamenetski~\cite{frank:dhe55}
rediscovered these results in his development of thermal explosion
theory.  Joseph and Lundgren~\cite{joseph:qdp73} gave an elementary
proof via phase plane analysis of the multiple existence of solutions
for dimensions $n\geq 3$.
\end{quote}
%\end{singlespace}
\par

% Several things to note here.  Latex sometimes breaks equations, this
% can be restricted by the samepage command.  The spacing in the array
% mode is also important for some structures.

From Zeidler~\cite{zeidler:nfa88IIa}:
{\samepage
\begin{equation}\label{station}
\begin{array}{rcll}
\dd\mbox{div } j& = &\dd f, &\dd x\in\Omega ,\\[8pt]
\dd u& = & \dd g_1, & \dd x\in\partial\Omega_1 , \\[8pt]
\dd j\nu & = & \dd g_2, & \dd x\in\partial\Omega_2 ,
\end{array}
\end{equation} }
where
\begin{equation}\label{current}
j =  h(|\nabla u|^2)\nabla u
\end{equation}
and $\Omega$ is a bounded domain in ${\Bbb R}^n$ with
smooth boundary $\partial\Omega = \overline{\partial\Omega_1}\cup
\overline{\partial\Omega_2}$, $\partial\Omega_1 \cap
\partial\Omega_2 = \emptyset$ and $\nu$ is the normal vector to
$\partial\Omega$.

% There is a lot of shorthand set up for structures, for example a
% lemma:
\begin{lemma}
Assuming that $\partial\Omega_2 = \emptyset$ and that $h(t) = 1$, we
have $$
\begin{array}{lr}
\dd\Delta u = f, & \dd x\in\Omega ,\\[8pt]
\dd u =  g_1, & \dd x\in\partial\Omega .
\end{array}
$$
\end{lemma}

% another ...
\begin{corollary}
If $g_2 = 0$ then
$$
\begin{array}{lr}
\dd \Delta u = f, & \dd x\in\Omega ,\\[8pt]
\dd u =  0, & \dd x\in\partial\Omega .
\end{array}
$$
\end{corollary}

% Look in thesis.sty for more structures.

\section{Fundamental results}
The investigation of the Gelfand problem begins with examining the
..... (this paragraph continues for many lines).

%
% This is an example of a big ugly technical theorem.  It has two
% levels of lists, referencing, citations and names.
%
\begin{theorem}[Joseph-Lundgren~\cite{joseph:qdp73}]
Boundary value problem (\ref{gelfand}) has positive radial
solutions $u$ on the unit ball which depend on $n$ and $\lambda$
in the following manner.
\begin{enumerate}
\item For $n=1,2$, there exists $\lambda^* >0$ such that
\begin{enumerate}
\item for $0< \lambda < \lambda^*$ there are two positive
solutions,
\item for $\lambda =\lambda^*$ there is a unique solution, and
\item for $\lambda > \lambda^*$ there are no solutions.
\end{enumerate}
\item For $3\leq n \leq 9$, let $\overline{\lambda}=2(n-2)$; then
there exist positive constants $\lambda_*$, $\lambda^*$ with
$0< \lambda_* < \overline{\lambda} < \lambda^*$, such that
\begin{enumerate}
\item for $\lambda = \lambda^*$ there is a unique solution,
\item for $\lambda > \lambda^*$ there are no solutions,
\item for $\lambda = \overline{\lambda}$ there is a countably infinite number
of solutions,
\item for $\lambda \in (\lambda_*,\lambda^*)$, $\lambda \neq
\overline{\lambda}$, there is a finite number of solutions,
\item for $\lambda < \lambda_*$ there is a unique solution.
\end{enumerate}
\item For $n\geq 10$, let $\lambda^* = 2(n-2)$ then
\begin{enumerate}
\item for $\lambda \geq \lambda^*$ there are no solutions,
\item for $\lambda \in (0,\lambda^*)$ there is a unique solution.
\end{enumerate}
\end{enumerate}
\end{theorem}
