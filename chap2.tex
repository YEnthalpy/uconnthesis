\chapter{Demonstration}\label{demonstration}   % level 1
\section{Introduction}
\label{1}

\indent Let $G$ be a finite group and let $k(G)$ be the
number of
its conjugacy classes or simple ${\bf C} [G]$-modules.  The
origin of this paper lies in a remarkable formula of
Kn\"orr and Robinson \citet{keller:btn69} which involves the numbers
$k(\ldots)$ for parabolic subgroups of a finite group $G$
with a split $(B,N)$ pair.  The problem posed and partially
solved in this paper concerns an analogous formula which
involves the numbers $k(\ldots)$ for parabolic subgroups of
a finite Coxeter group $W$.  The work here does not depend
on \\citet{keller:btn69} and has, at least for the present, no direct
contact with it.  Nevertheless it seems appropriate to
state some of the results of \citet{keller:btn69} in this
Introduction.

The main results of \citet{keller:btn69} give a reformulation, in
homological terms, of a conjecture of Alperin \citet{arnold:gmt83}
about the representations of any finite group over an
algebraically closed field $K$ of characteristic $p >
0$, as well as certain applications to groups where the
conjecture is known to be true.  Alperin's conjecture
states that the number of isomorphism classes of simple
$K[G]$-modules is equal to the number of weights for $G$.
A weight for $G$ is, by definition, a pair $(Q,S)$ where
$Q$ is a $p$-subgroup and $S$ is a simple $K[N(Q)]$-module
which is projective when regarded as a module for
$K[N(Q)/Q]$; two weights are considered to be the same if
the subgroups are conjugate and the modules are isomorphic \citep{amann:gd83}.

Suppose now that $G$ has a split $(B,N)$ pair and that $p$
is the natural characteristic for $G$.  Cabanes \citet{collett:imi80}
has shown that Alperin's conjecture is true in this case.
Let $W$ be the Weyl group and let $I$ be a set of Coxeter
generators for $W$.  If $J \subseteq I$ let $W_{J}$ be the
corresponding parabolic subgroup of $W$ and let $P_{J} =
BW_{J}B$ be the corresponding parabolic subgroup of $G$.
Let $l(P_{J})$ be the number of $p$-regular conjugacy
classes or simple $K[P_{J}]$-modules.  Kn\"orr and
Robinson show using their general results, Cabanes' theorem
and the homology of the Tits building that
\begin{equation}
 \sum_{J \subseteq I} (-1)^{|I-J|} l(P_{J}) = f_{0}(G)
\label{101}
\end{equation}
and also
\begin{equation}
 \sum_{J \subseteq I} (-1)^{|I-J|} k(P_{J}) = f_{0}(G)
\label{102}
\end{equation}
where $f_{0}(G)$ is the number of $p$-blocks of $G$ of
defect zero.
